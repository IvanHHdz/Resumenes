\section{Álgebra}

\subsection{Propiedades de los números reales}

Algunas de las propiedades más importantes de los números reales.

\subsubsection{Exponentes}
\begin{align*}
    x^0 &= 1 \quad \text{si } x \neq 0 \\
    x^1 &= x \\
    x^{-1} &= \frac{1}{x} \quad \text{si } x \neq 0 \\
    x^{m + n} &= x^m x^n \\
    x^{m - n} &= \frac{x^m}{x^n} \quad \text{si } x \neq 0 \\
    x^{-1} &= \frac{1}{x} \\
    x^{mn} &= {(x^m)}^n \\
    x^{\frac{m}{n}} &= \sqrt[n]{x^m} = {\left(\sqrt[n]{x}\right)}^m \\
    x^{\frac{1}{n}} &= \sqrt[n]{x}
\end{align*}

\subsubsection{Logaritmos}
Recordemos que el logaritmo se define como:
\[
    \log_a(b) = c \iff a^c = b
\]
Y sus propiedades más importantes son:
\begin{align*}
    \log_a(1) &= 0 \\
    \log_a(a^u) &= u \\
    a^{\log_a(u)} &= u \\
    \log_a(a) &= 1 \\
    \log_a(uv) &= \log_a(u) + \log_a(v) \\
    \log_a\left(\frac{u}{v}\right) &= \log_a(u) - \log_a(v) \\
    \log_a(u^n) &= n \log_a(u) \\
    \log_a(b) &= \frac{\log_c(b)}{\log_c(a)}
\end{align*}

\subsection{Factorización}
Algunas fórmulas de factorización importantes.
\begin{align*}
    na + nb &= n(a + b) \\
    a^2 + b^2 &=  {(a + b)}^2 - 2ab = {(a - b)}^2 + 2ab\\
    a^2 - b^2 &= (a - b)(a + b) \\
    a^3 + b^3 &= (a + b)(a^2 - ab + b^2) \\
    a^3 - b^3 &= (a - b)(a^2 + ab + b^2) \\
\end{align*}

\subsubsection{Completación de cuadrados}
Para reescribir una expresión cuadrática de la forma \(ax^2 + bx + c\) como un cuadrado perfecto, se utiliza la fórmula:
\[
    ax^2 + bx + c = a{\left(x + \frac{b}{2a}\right)}^2 - \frac{b^2 - 4ac}{4a}
\]

\subsection{Sumatorias}
Consiste en una forma compacta de expresar la suma de una secuencia de números. Se denota como:
\[
    \sum_{i = 1}^n a_i = a_1 + a_2 + \cdots + a_n
\]
\subsubsection{Propiedades}
\begin{align*}
    \sum_{i = 1}^n c a_i &= c \sum_{i = 1}^n a_i \\
    \sum_{i = 1}^n(a_i \pm b_i) &= \sum_{i = 1}^n a_i \pm \sum_{i = 1}^n b_i \\
    \sum_{i = 1}^n a_i &= \sum_{i = 1}^m a_i + \sum_{i = m + 1}^n a_i
\end{align*}

\subsubsection{Fórmulas de sumatorias}
Algunas fórmulas de sumatorias importantes son:
\begin{align*}
    \sum_{i = 1}^n i &= \frac{n(n + 1)}{2} \\
    \sum_{i = 1}^n i^2 &= \frac{n(n + 1)(2n + 1)}{6} \\
    \sum_{i = 1}^n i^3 &= {\left(\frac{n(n + 1)}{2}\right)}^2 \\
\end{align*}

\subsection{Teorema generalizado del binomio de Newton}
El teorema generalizado del binomio de Newton establece que para cualquier número real o complejo \(n\) donde \(|y| < |x|\) se tiene que:
\[    
{(x + y)}^n = \sum_{k = 0}^{\infty} \binom{n}{k} x^{n - k} y^k 
\]
donde \(\binom{n}{k}\) es el coeficiente binomial, que se define como:
\[
    \binom{n}{k} = \frac{n(n - 1)(n - 2) \cdots (n - k + 1)}{k!}
\]
Y si \(n\in\mathbb{N}\) entonces podemos reescribir todo como:
\[    
{(x + y)}^n = \sum_{k = 0}^{n} \binom{n}{k} x^{n - k} y^k 
\]
Y el coeficiente binomial se define como:
\[
    \binom{n}{k} = \frac{n!}{k!(n - k)!}
\]

\subsection{Fracciones parciales}
Las fracciones parciales permiten descomponer una fracción racional en una suma de fracciones más simples.  
Este método se usa principalmente para integrar funciones racionales, es decir, cocientes de polinomios.

Dada una fracción racional:
\[
\frac{P(x)}{Q(x)}
\]
donde \(P(x)\) y \(Q(x)\) son polinomios, se desea expresar como suma de fracciones más simples que se puedan integrar fácilmente.

Si el grado del numerador es mayor o igual al del denominador:
\[
\deg P(x) \geq \deg Q(x)
\]
entonces se realiza la división polinómica:
\[
\frac{P(x)}{Q(x)} = S(x) + \frac{R(x)}{Q(x)}
\]
donde \(S(x)\) es el cociente y \(R(x)\) el residuo.  
Luego se trabaja solo con la parte fraccionaria \(\frac{R(x)}{Q(x)}\).

Se descompone el polinomio \(Q(x)\) en factores lineales y/o cuadráticos irreducibles, por ejemplo:
\[
Q(x) = {(x - a)}^m {(x^2 + bx + c)}^n \cdots
\]

Dependiendo del tipo de factor en el denominador:

\begin{itemize}
    \item \textbf{Factor lineal simple} \((x - a)\): 
    \[
    \frac{A}{x - a}
    \]
    
    \item \textbf{Factor lineal repetido} \({(x - a)}^n\): 
    \[
    \frac{A_1}{x - a} + \frac{A_2}{{(x - a)}^2} + \cdots + \frac{A_n}{{(x - a)}^n}
    \]

    \item \textbf{Factor cuadrático irreducible} \((x^2 + bx + c)\): 
    \[
    \frac{Ax + B}{x^2 + bx + c}
    \]

    \item \textbf{Factor cuadrático repetido} \({(x^2 + bx + c)}^n\): 
    \[
    \frac{A_1x + B_1}{x^2 + bx + c} + \frac{A_2x + B_2}{{(x^2 + bx + c)}^2} + \cdots + \frac{A_n x + B_n}{{(x^2 + bx + c)}^n}
    \]
\end{itemize}

Pueden usarse factores de mayores grados, pero la idea es que cada término de la fracción parcial sea más simple que el original.

Multiplicamos ambos lados por \(Q(x)\) para eliminar los denominadores:
\[
P(x) = \text{(expresión con constantes)}
\]

\begin{itemize}
    \item \textbf{Sustitución}: elige valores de \(x\) que simplifiquen la ecuación.
    \item \textbf{Comparación de coeficientes}: iguala los coeficientes de cada potencia de \(x\).
\end{itemize}