\section{Cálculo Multivariable}
\subsection{Derivadas parciales}

Son las derivadas de funciones de varias variables. 
Se denotan por:
\[
\frac{\partial f}{\partial x} \quad \text{y} \quad \frac{\partial f}{\partial y}
\]

Estas se leería como ``derivada parcial de \(f\) con respecto a \(x\)'' y ``derivada parcial de \(f\) con respecto a \(y\)'', respectivamente.

Cuando se deriva una función con varias variables, se debe hacer con respecto a una sola de las variables
Al calcular la derivada con respecto a una variable, el resto se consideran como constantes.

\subsection{Derivada Total}
Considere una función de varias variables que llamaremos \(z\).
La derivada total de \(z\), denotada por \(dz\), es la suma de las derivadas parciales de \(z\) con respecto a cada variable, multiplicadas por el diferencial de esa variable.

Por ejemplo, si \(z\) es una función de \(x\) e \(y\), la derivada total se expresa como:
\[
dz = \frac{\partial z}{\partial x} dx + \frac{\partial z}{\partial y} dy
\]

\subsubsection{Regla de la cadena}
La regla de la cadena se utiliza para calcular la derivada de una función compuesta.
Si \(z\) es una función de \(x\) e \(y\), y \(x\) e \(y\) son funciones de \(t\), entonces la derivada total de \(z\) es:
\[
dz = \frac{\partial z}{\partial x} \frac{dx}{dt} + \frac{\partial z}{\partial y} \frac{dy}{dt}
\]

Y si en lugar de depender de \(t\), \(y\) depende de \(u\) y de \(v\), la expresión se convierte en:

\[
dz = \frac{\partial z}{\partial x} \frac{d x}{d t} dt  + \frac{\partial z}{\partial y}\left(\frac{\partial y}{\partial u} du + \frac{\partial y}{\partial v} dv\right)
\]

Y así sucesivamente, dependiendo de cuántas variables dependan de otras.

Y si queremos la derivada de \(z\) con respecto a \(t\), podemos escribir:
\[
\frac{dz}{dt} = \frac{\partial z}{\partial x} \frac{dx}{dt} + \frac{\partial z}{\partial y} \frac{dy}{dt}
\]

\subsection{Gradiente}
El gradiente de una función \(f(x, y)\) es un vector que contiene las derivadas parciales de \(f\) con respecto a cada variable. Se denota como:
\[
\nabla f = \left( \frac{\partial f}{\partial x}, \frac{\partial f}{\partial y} \right)
\]

Si \(f\) es una función de varias variables, el gradiente se extiende a más dimensiones:
\[
\nabla f = \left( \frac{\partial f}{\partial x_1}, \frac{\partial f}{\partial x_2}, \ldots, \frac{\partial f}{\partial x_n} \right)
\]

\subsection{Derivada direccional}
La derivada direccional de una función \(f\) en la dirección de un vector unitario \(\vec{u}\) se define como:
\[
D_{\vec{u}} f = \nabla f \cdot \vec{u}
\]

Donde \(\nabla f\) es el gradiente de \(f\) y \(\cdot\) representa el producto punto.

Es importante que el vector \(\vec{u}\) sea unitario, es decir, que su norma sea 1. 
Si \(\vec{u}\) no es unitario, se puede normalizar dividiendo por su norma.

\subsection{Integrales dobles}
Las Integrales dobles o integrales iteradas son una extensión de la integral definida a funciones de varias variables.
Para una función \(f(x, y)\) definida en un dominio \(D\), puede ser de tipo I o II.\ 
\[
\int_a^b \int_{g_1(x)}^{g_2(x)} f(x,y)\,dy\,dx
\]

Siendo esta primera la de tipo I, donde \(g_1(x)\) y \(g_2(x)\) son funciones que definen los límites de integración en \(y\) para cada \(x\) en el intervalo \([a, b]\).

Y la segunda de tipo II, donde \(h_1(y)\) y \(h_2(y)\) son funciones que definen los límites de integración en \(x\) para cada \(y\) en el intervalo \([c, d]\):
\[
\int_c^d \int_{h_1(y)}^{h_2(y)} f(x,y)\,dx\,dy
\]

Para calcular una integral doble, se integra primero con respecto a una variable y luego con respecto a la otra.
Mientras se integra con respecto a una variable, la otra se considera constante.
El orden de integración puede cambiar dependiendo de la función y del dominio de integración.

\subsubsection{Teorema de Fubini}
El teorema de Fubini establece que, bajo ciertas condiciones de continuidad, el orden de integración en una integral doble puede cambiar. 
Es decir:
\[
\iint_R f(x,y) \, dA =\int_a^b \int_{g_1(x)}^{g_2(x)} f(x,y)\,dy\,dx=\int_c^d \int_{h_1(y)}^{h_2(y)} f(x,y)\,dx\,dy
\]

Donde \(R\) es el dominio de integración y \(dA\) representa el elemento diferencial de área, que se puede expresar como \(dA = dx\,dy\) o \(dA = dy\,dx\), dependiendo del orden de integración.

\subsection{Integrales triples}
Las integrales triples son una extensión de las integrales dobles a funciones de tres variables.
Para una función \(f(x, y, z)\) definida en un dominio \(D\), la integral triple se expresa como:
\[
\iiint_D f(x, y, z) \, dV
\]

Donde \(dV\) es el elemento diferencial de volumen, que puede expresarse como
\(dV = dx\,dy\,dz\), \(dV = dy\,dz\,dx\), o cualquier otra combinación de variables.

Para el cálculo de volúmenes, se pueden utilizar coordenadas cartesianas, cilíndricas o esféricas, dependiendo de la simetría del dominio.
Además de poderse considerar que \(f(x, y, z) = 1\) y pasar la integral doble a una integral triple, para calcular el volumen del dominio \(D\):
\[
V = \iint _R f(x,y) \,dA = \iiint_D f(x,y,z)\,dV
\]

\subsection{Cambios de coordenadas}
Los cambios de coordenadas son útiles para simplificar el cálculo de integrales en dominios complicados.

\subsubsection{Integrales dobles}
Se suelen cambiar de coordenadas cartesianas (o rectangulares) a coordenadas polares, donde:
\begin{align*}
x &= r \cos(\theta) \\
y &= r \sin(\theta)
\end{align*}

Y viceversa:
\begin{align*}
r &= \sqrt{x^2 + y^2} \\
\theta &= \tan^{-1}\left(\frac{y}{x}\right)
\end{align*}

El elemento diferencial de área:
\[
dA = \underbrace{dx \, dy}_{\text{cartesianas}} = \underbrace{r \, dr \, d\theta}_{\text{polares}}
\]

\subsubsection{Integrales triples}
Para integrales triples, se pueden utilizar coordenadas cartesianas (o rectangulares), cilíndricas o esféricas.

En coordenadas cilíndricas, las relaciones son:
\begin{align*}
x &= r \cos(\theta) \\
y &= r \sin(\theta) \\
z &= z
\end{align*}
Y viceversa:
\begin{align*}
r &= \sqrt{x^2 + y^2} \\
\theta &= \tan^{-1}\left(\frac{y}{x}\right) \\
z &= z
\end{align*}

En coordenadas esféricas, las relaciones son:
\begin{align*}
x &= \rho \sin(\phi) \cos(\theta) \\
y &= \rho \sin(\phi) \sin(\theta) \\
z &= \rho \cos(\phi)
\end{align*}
Y viceversa:
\begin{align*}
\rho &= \sqrt{x^2 + y^2 + z^2} \\
\phi &= \cos^{-1}\left(\frac{z}{\sqrt{x^2 + y^2 + z^2}}\right) \\
\theta &= \tan^{-1}\left(\frac{y}{x}\right)
\end{align*}

Y el elemento diferencial de volumen:
\[
dV = \underbrace{dx \, dy \, dz}_{\text{cartesianas}} = \underbrace{r \, dr \, d\theta \, dz}_{\text{cilíndricas}} = \underbrace{\rho^2 \sin(\phi) \, d\rho \, d\phi \, d\theta}_{\text{esféricas}}
\]

\subsection{Cálculo de áreas y volúmenes}
Para calcular áreas y volúmenes, se utilizan integrales dobles y triples, respectivamente.
Para el área de una región \(R\) en el plano, se utiliza la integral doble:
\[
A = \iint_R dA
\]
Para el volumen de un sólido \(D\) en el espacio, se utiliza la integral triple:
\[
V = \iiint_D dV
\]

En estos casos \(R\) y \(D\) son los dominios de integración en el plano y en el espacio, respectivamente.
Lo que quiere decir que representan regiones o sólidos normalmente delimitados por funciones o superficies.
