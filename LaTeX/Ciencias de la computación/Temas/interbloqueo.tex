\section{Demostración de Prevención de Interbloqueo mediante un Algoritmo de Asignación Atómica}
\subsection{Planteamiento}

Sea \( R \) el conjunto de recursos de un sistema de computación multiprocesado y multiprogramado, y sea \( H \) el conjunto de procesos (o hilos) que pueden ejecutarse en el sistema.

Definimos un algoritmo \( A \) con las siguientes propiedades:

\begin{enumerate}
    \item Todo proceso debe pasar por \( A \) para solicitar recursos.
    \item \( A \) verifica, de forma atómica, si todos los recursos requeridos por el proceso están disponibles.
    \item Si están disponibles, se le conceden todos a la vez.
    \item Si no están disponibles, el proceso espera sin obtener ningún recurso.
    \item Si un proceso necesita más recursos después, debe liberar todos los que ya posee y volver a solicitarlos desde cero a través de \( A \).
    \item \( A \) actúa como un \textbf{monitor}: solo un proceso puede estar dentro de él a la vez.
\end{enumerate}

\bigskip

Queremos demostrar que con este algoritmo es \textbf{imposible que ocurra un interbloqueo (deadlock)}.

\subsection{Demostración por inducción sobre el número de procesos}

\subsubsection{Caso base: \( P(1) \)}

Supongamos que hay un único proceso \( p_1 \) que solicita un conjunto de recursos.

Como es el único proceso en el sistema, todos los recursos están disponibles. Entonces, el algoritmo \( A \) le asigna todos los recursos necesarios.

\textbf{Conclusión:} No hay espera ni bloqueo. El sistema no presenta interbloqueo.

\subsubsection{Hipótesis inductiva: \( P(k) \)}

Supongamos que con \( k \) procesos en ejecución, utilizando el algoritmo \( A \), no se produce interbloqueo.

\subsubsection{Paso inductivo: \( P(k+1) \)}

Agregamos un proceso más: \( p_{k+1} \). Este entra al algoritmo \( A \) para solicitar un conjunto de recursos.

\begin{itemize}
    \item Si todos los recursos que necesita están disponibles, se le conceden de inmediato. No hay bloqueo.
    \item Si alguno no está disponible, no se le concede ninguno, y espera fuera del monitor sin retener recursos.
\end{itemize}

Además, si otro proceso \( p_c \) quiere recursos adicionales, debe liberarlos todos antes de hacer una nueva solicitud, lo cual evita la retención y espera.

Esto impide la formación de cadenas de espera circulares, condición necesaria para que ocurra un interbloqueo.

\textbf{Conclusión:} Al agregar un proceso más, el sistema sigue libre de interbloqueo.

\subsubsection{Conclusión final}

Por el principio de inducción matemática, se concluye que:

\begin{center}
\textbf{Ningún conjunto finito de procesos que usen el algoritmo \( A \) puede entrar en interbloqueo.}
\end{center}